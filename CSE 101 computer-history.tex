%********************************************************************************
%       Preamble Information
%********************************************************************************
\documentclass[11pt, twocolumn]{article}
\usepackage[T1]{fontenc}
\usepackage[utf8]{inputenc}
\usepackage{mathpazo}
\usepackage{multicol}
\setlength\columnsep{20pt}

\usepackage{hyperref}
\hypersetup{
    colorlinks=true,
    linkcolor=blue,
    filecolor=magenta,      
    urlcolor=blue,
    citecolor=black
    }
% Used to include graphic images into LaTeX
\usepackage{graphicx}
% Tells the compiler to search in the images/ 
% folder for any included figures
\graphicspath{ {./images/} }

% Make lists more compact
% (from pandoc template)
\providecommand{\tightlist}{\setlength{\itemsep}{0pt}\setlength{\parskip}{0pt}}

% Prevent overfull lines
\setlength{\emergencystretch}{1em}

% No indentation and space between paragraph
\setlength{\parindent}{0.0in}
%\setlength{\parskip}{0.04in}
% Have space between paragraph
\setlength{\parskip}{11pt}

% Margin support
\usepackage[margin=0.75in]{geometry}

% header and footer
\usepackage{fancyhdr}

\newcommand{\thelabnumber}{HW3}
\newcommand{\thetitle}{HW3: Computer History}
% Update this line with your name 
\newcommand{\theauthor}{Jared Martinez \\ Logan Baeza}

% Write title and author
\title{{\large } \thetitle}
\author{\theauthor}
\date{\today}

\pagestyle{fancy}
\fancyhf{}
\fancyfoot[R]{\thepage}
\fancyfoot[C]{HW3}
\fancyfoot[L]{CSE/IT 101 HW3}
\renewcommand{\footrulewidth}{1pt}
\renewcommand{\headrulewidth}{0pt}
\fancypagestyle{firstpage}{%
  \fancyhf{}% clear default for header and footer
  \fancyfoot[R]{\thepage}
  \fancyfoot[C]{HW3}
  \fancyfoot[L]{CSE/IT 101}
  \renewcommand{\footrulewidth}{1pt}
    \renewcommand{\headrulewidth}{0pt}
}

%********************************************************************************
%      Begin Document
%********************************************************************************
\begin{document}
\maketitle

\thispagestyle{firstpage}

%********************************************************************************
%      Report Content
%********************************************************************************
\section{Introduction}
Supercomputers are some of the most intricate, complex, and utterly awe-inspiring technological marvels that human kind has ever created. With the ability to preform millions of calculations in a matter of seconds, these machines have greatly advanced the human understanding of science. Using multiple powerful CPU's, a supercomputer sends signals within itself at almost the speed of light, with the fastest performing a massive 442 quadrillion (\begin{math}4.42x10^1^5\end{math}) Floating-Point-Operations per Second (FLOPS). By comparison, a standard household computer can only process at a speed of 2.8 billion (\begin{math}2.8x10^9\end{math}) Operations per Second.

\section{Time Period}
The concept of a supercomputer has existed for a remarkably long time in relation to computer history. In 1964, Seymoore Cray was credited for building the Control Data Corporation's CDC 6600, after a brief competition with IBM to build an extremely fast computer. This was the first computer to be dubbed as a supercomputer, with the ability to perform three million FLOPS. After 57 years of improvement, the official record holder for the world's fastest supercomputer is Fujitsu Global's Fugaku Supercomputer, which was launched in 2021, with a normal processing speed of 442 quadrillion (\begin{math}4.42x10^1^5\end{math}) Floating-Point-Operations per Second (FLOPS), with the ability to be overclocked to be faster. Research still continues to this day as to how to increase the speed of these computers, but the processes are being controlled by the limitation that data cannot be transferred at a speed above the speed of light.

\section{Computer Hardware}
Supercomputers use the same general hardware as a standard computer, but on a much larger, more powerful scale. Many supercomputers are the size of a whole room, with large cases containing hundreds of thousands of CPU's, GPU's, memory banks, and user peripherals. Supercomputers need this much hardware in order to reach their maximum processing speeds. The Fugaku Supercomputer has a total memory capacity of 4.85 PiB, or 5460615 gB of memory, across three layers of storage systems, and 48 cores dedicated to processing, with 6 additional cores for support. All of this technology did not come cheap, with a total price tag of roughly \$1 Billion USD 

Here, Figure \ref{fig:gpu} has been included so you have some code to use and update.
\begin{figure}
    \centering
    \includegraphics[width=0.45\textwidth]{gpu}
    \caption{Intel Arc GPU}
    \label{fig:gpu}
\end{figure}

I found this image on the PC World webpage \cite{Ung21}. 

\subsection{Computer Software}
cats n' dogs

\subsection{Conclusion}
Conclude your research paper with any reflections on what you learned about your 
topic. Was this what you expected to find? Did you find any facts that surprised you?
You may add other personal reflections about the topic here.

%********************************************************************************
%      Report Content
%********************************************************************************
\bibliographystyle{acm}
\bibliography{sources}


\end{document}
